%-------------------
\usepackage[utf8]{inputenc}
\usepackage[T1]{fontenc}
\usepackage[frenchb]{babel}
%\usepackage[english]{babel}

%-------------------
\usepackage{minted} %pour inclure du code
\setminted[ocaml]{	                               
	%bgcolor = black!3!white,                           %--fond 
	frame = leftline,	framesep = 6pt, rulecolor= expli, %-- cadre
	linenos=true, numbersep=4pt, xleftmargin=20pt,      %--numéro de ligne
	breaklines=true,                                    %--découpage des longues lignes
	tabsize=2,     %--tabulations    
	fontsize=\scriptsize,
}

\setminted[c]{	                               
	%bgcolor = black!3!white,                           %--fond 
	frame = leftline,	framesep = 6pt, rulecolor= expli, %-- cadre
	linenos=true, numbersep=4pt, xleftmargin=20pt,      %--numéro de ligne
	breaklines=true,                                    %--découpage des longues lignes
	tabsize=2,                                          %--tabulations
}

%-------------------
\usepackage[ruled,vlined]{algorithm2e} %pour le pseudo-code
%ajouter l'option linesnumbered si on veut les numéros de lignes
\SetAlgorithmName{Algorithme}{Algo}{Liste des algorithmes}
\SetFuncArgSty{textup}%style des arguments des fonctions pas en gras
\SetArgSty{textup}%style des condidtions des if, for, while pas en gras


\SetKwFor{ForEach}{pour tout}{faire}{}
\SetKwFor{For}{pour}{faire}{finpour}
\SetKwIF{Si}{SinonSi}{Sinon}{si}{alors}{sinon si}{sinon}{}
\SetKwInput{Input}{Entrée}
\SetKwInput{Output}{Sortie}
\SetKwProg{myproc}{Procedure}{:}{}
\SetKw{Return}{retourner}
\SetKwComment{tcc}{(*}{*)}
\SetKwFor{Tq}{tant que}{faire}{}
\SetKwRepeat{Repeter}{répéter}{jusqu’à}

%-------------------
\usepackage{amsmath,amsfonts,amssymb}
\usepackage{textcomp,lmodern}%pour l'euro
\usepackage{cancel}%pour barrer

%-------------------
\usepackage{array,multirow}
\usepackage{tabularx}%tableaux élastiques



%-------------------
\usepackage{tikz} %graphiques et dessins

\usetikzlibrary {fit}
\usetikzlibrary{shapes}%pr écrire \node[ellipse] par ex
\usetikzlibrary{positioning} %pr écrire \node[below rigth = 3pt and 5pt]
\usetikzlibrary{decorations.pathreplacing}%pr les accolades
\usetikzlibrary{patterns}%pour les hachures
\usetikzlibrary{automata} % pour les automates


%-------------------
\usepackage{caption}%les personaliser (les centrer par ex)
\usepackage{changepage}%pour élargir la page avec adjustwidth
\usepackage{calc}%pour pouvoir écrire \textwidth-1cm


%-------------------
\usepackage{xspace}%espaces après les commandes texte


%------------------

%-------------------



\usepackage[clock]{ifsym}
