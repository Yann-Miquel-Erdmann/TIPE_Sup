\subsection{Analyse lexicale}

\begin{frame}
    
   \begin{tikzpicture}
    
    \tikzstyle{lien} = [->, >=latex]
    \tikzstyle{basic_text}=[text width=2cm, text badly centered]
    \tikzstyle{basic_node}=[draw = black,rounded corners=4pt, basic_text]
    \tikzstyle{wrapper}=[basic_node, inner sep=3pt]
    \tikzstyle{hidden}=[draw=black!0,color=black!0]
    \tikzstyle{faded}=[draw=black!20, color=black!20]
    

    \node [basic_node, text width=5cm, align=left] (Fortran) {
        \scalebox{0.8}{
            \parbox{\textwidth}{
                \textbf{Programme Fortran}
                \inputminted{fortran}{static/HelloWorld.f90}
            }
        }
    };

    \node [basic_node, text width=5cm, align=left] (Lexemes) [right=1cm of Fortran] {
        \scalebox{0.8}{
            \parbox{\textwidth}{

                \textbf{Liste de lexèmes}
                \inputminted{ocaml}{static/lexemes.ml}

            }
        }
    };
 
    \draw [lien] (Fortran.north) to[out=45, in=135]  node[midway, above] {\textbf{Analyse lexicale}}   (Lexemes.west) ; 

\end{tikzpicture}

\end{frame}

\begin{frame}
  \frametitle{Définition d'un automate\esp}

  
  Les automates sont définis par $(\mathcal{A} = \Sigma, Q, I, F, \delta)$\\

  \vspace{0.5cm}
  \pause
  \begin{tikzpicture}
    \node [state, initial]         (1) at (0, 0)        {1};
    \node [state]                  (2) [right=1cm of 1] {2};
    \node [state, accepting right] (3) [right=1cm of 2] {3};

    \path [->]
    (1) edge              node [above] {$a$} (2)
    (2) edge              node [above] {$c$} (3)
    (2) edge [loop above] node [above] {$b$} (2)
    ;

  \end{tikzpicture}\\
  $\mathcal{L} (\mathcal{A} ) = a\cdot b^*\cdot c$

  
\end{frame}

%------------étude d'un exemple-----------
\begin{frame}
  \frametitle{Les automates: étude d'un exemple \esp}
    \begin{tikzpicture}
    \node at (3, 0) (text) {\ttfamily print*,"Hello World"};
    \begin{scope}
      [every node/.style={scale=0.6,font=\Large}, scale=0.4, initial text=, shift={(-13, 0)}]
      \tikzstyle{style0}=[]
      \tikzstyle{style1}=[]
      \tikzstyle{style2}=[]
      \tikzstyle{style3}=[]
      \tikzstyle{style4}=[] 
      \tikzstyle{style5}=[]
      \tikzstyle{style6}=[]
      \tikzstyle{style7}=[]
      \tikzstyle{style8}=[]
      \tikzstyle{style9}=[]

      \only<1> {\tikzstyle{style0}=[draw=blue!50,very thick,fill=blue!20]}
      \only<2> {\tikzstyle{style1}=[draw=blue!50,very thick,fill=blue!20]}
      \only<3> {\tikzstyle{style2}=[draw=blue!50,very thick,fill=blue!20]}
      \only<4> {\tikzstyle{style3}=[draw=blue!50,very thick,fill=blue!20]}
      \only<5> {\tikzstyle{style4}=[draw=blue!50,very thick,fill=blue!20]}
      \only<6> {\tikzstyle{style5}=[draw=blue!50,very thick,fill=blue!20]}
      \only<7> {\tikzstyle{style0}=[draw=blue!50,very thick,fill=blue!20]}
      \only<8> {\tikzstyle{style8}=[draw=blue!50,very thick,fill=blue!20]}
      \only<9> {\tikzstyle{style0}=[draw=blue!50,very thick,fill=blue!20]}
      \only<10>{\tikzstyle{style9}=[draw=blue!50,very thick,fill=blue!20]}
      \only<11> {\tikzstyle{style0}=[draw=blue!50,very thick,fill=blue!20]}
      \only<12-23>{\tikzstyle{style6}=[draw=blue!50,very thick,fill=blue!20]}
      \only<24>{\tikzstyle{style7}=[draw=blue!50,very thick,fill=blue!20]}

      \node[state, style0, initial        ]  at (0,0) (0)          {$I$};

      \node[state, style1] (1) at (2,1) {1};
      \node[state, style2] (2) [right=0.4cm of 1] {2};
      \node[state, style3] (3) [right=0.4cm of 2] {3};
      \node[state, style4] (4) [right=0.4cm of 3] {4};
      \node[state, accepting right, style5] (5) [right=0.4cm of 4] {$F_1$};
      
      \node[state, style6,                ] (6) at (2,3) {5};
      \node[state, style7, accepting right] (7) [right=0.4cm of 6] {$F_2$};
      
      \node[state, style8, accepting right] (8) at (2,-1)      {$F_3$};
      \node[state, style9, accepting right] (9) at (2,-3)      {$F_4$};

      \path[->]
      (0) edge                node[above] {$*$} (8)
      (0) edge                node[above] {\LARGE$,$} (9)

      (0) edge                node[above] {$p$} (1)
      (1) edge                node[above] {$r$} (2)
      (2) edge                node[above] {$i$} (3)
      (3) edge                node[above] {$n$} (4)
      (4) edge                node[above] {$t$} (5)
      
      
      (0) edge                node[above] {$"$} (6)
      (6) edge [loop above]   node[above] {$[a-z]|[A-Z]$} (6)
      (6) edge                node[above] {$"$} (7)
      ;
      
    \end{scope}
    \foreach \step in {1,2,...,6}{
      \node<\step>[right=\fpeval{\step * 5.75 - 3}pt of text.west, draw=blue, fill=blue!20, inner sep=0pt, minimum width=2pt, minimum height=10pt]{};
    }
    \foreach \step in {7,8}{
      \node<\step>[right=\fpeval{(\step - 1) * 5.75 - 3}pt of text.west, draw=blue, fill=blue!20, inner sep=0pt, minimum width=2pt, minimum height=10pt]{};
    }
    \foreach \step in {9,10}{
      \node<\step>[right=\fpeval{(\step - 2) * 5.75 - 3}pt of text.west, draw=blue, fill=blue!20, inner sep=0pt, minimum width=2pt, minimum height=10pt]{};
    }
    \foreach \step in {11,12,...,24}{
      \node<\step>[right=\fpeval{(\step - 3) * 5.75 - 3}pt of text.west, draw=blue, fill=blue!20, inner sep=0pt, minimum width=2pt, minimum height=10pt]{};
    }

    \node at (0, -2.3) (lb) {
      \begin{minipage}{\textwidth}
        \only<1-6>  {\inputminted[firstline=1, lastline=1, linenos=false, frame=none, xleftmargin=0pt,]{ocaml}{static/exampleList.ml}}
        \only<7-8>  {\inputminted[firstline=2, lastline=2, linenos=false, frame=none, xleftmargin=0pt,]{ocaml}{static/exampleList.ml}}
        \only<9-10> {\inputminted[firstline=3, lastline=3, linenos=false, frame=none, xleftmargin=0pt,]{ocaml}{static/exampleList.ml}}
        \only<11-24>{\inputminted[firstline=4, lastline=4, linenos=false, frame=none, xleftmargin=0pt,]{ocaml}{static/exampleList.ml}}
        \only<25>   {\inputminted[firstline=5, lastline=5, linenos=false, frame=none, xleftmargin=0pt,]{ocaml}{static/exampleList.ml}}
      \end{minipage}
    };
  \end{tikzpicture}
\end{frame}


%------------étude d'un exemple-----------
\begin{frame}
  \frametitle{Les automates: étude d'un exemple\esp}
  \begin{tikzpicture}
    \node at (1, 0) (text1) {\ttfamily print};
    \node [right=-5pt of text1.east, anchor=west] (text2) {\ttfamily print};
    \node [right=-5pt of text2.east, anchor=west] (text3) {\ttfamily "Hello World"};

    \begin{scope}
      [every node/.style={scale=0.6,font=\Large}, scale=0.4, initial text=, shift={(-13, 0)}]
      \tikzstyle{style0}=[]
      \tikzstyle{style1}=[]
      \tikzstyle{style2}=[]

      \only<1> {\tikzstyle{style0}=[draw=blue!50,very thick,fill=blue!20]}
      \only<2-3>{\tikzstyle{style1}=[draw=blue!50,very thick,fill=blue!20]}
      \only<4> {\tikzstyle{style2}=[draw=blue!50,very thick,fill=blue!20]}

      \node[state, style0, style1, style2, initial]  at (0,0) (0)          {$I$};

      \node[state, style1] (1) at (2,1) {1};
      \node[state, style1] (2) [right=0.4cm of 1] {2};
      \node[state, style1] (3) [right=0.4cm of 2] {3};
      \node[state, style1] (4) [right=0.4cm of 3] {4};
      \node[state, accepting right, style1] (5) [right=0.4cm of 4] {$F_1$};
      
      \node[state, style2,                ] (6) at (2,3) {5};
      \node[state, style2, accepting right] (7) [right=0.4cm of 6] {$F_2$};
      
      \node[state, accepting right]         (8) at (2,-1)      {$F_3$};
      \node[state, accepting right]         (9) at (2,-3)      {$F_4$};

      \path[->]
      (0) edge                      node[above] {$*$} (8)
      (0) edge                      node[above] {\LARGE$,$} (9)

      (0) edge [style1]             node[above] {$p$} (1)
      (1) edge [style1]             node[above] {$r$} (2)
      (2) edge [style1]             node[above] {$i$} (3)
      (3) edge [style1]             node[above] {$n$} (4)
      (4) edge [style1]             node[above] {$t$} (5)
      
      
      (0) edge [style2]             node[above] {$"$} (6)
      (6) edge [loop above, style2] node[above] {$[a-z]|[A-Z]$} (6)
      (6) edge [style2]             node[above] {$"$} (7)
      ;
      
    \end{scope}
    
    \node<2>[fit=(text1), draw=blue, fill=blue!20, inner sep=-2pt, fill opacity=0.5]{};
    \node<3>[fit=(text2), draw=blue, fill=blue!20, inner sep=-2pt, fill opacity=0.5]{};
    \node<4>[fit=(text3), draw=blue, fill=blue!20, inner sep=-2pt, fill opacity=0.5]{};
    \node at (0, -2.3) {
      \begin{minipage}{\textwidth}
        \only<1>{\inputminted[firstline=1, lastline=1, linenos=false, frame=none, xleftmargin=0pt,]{ocaml}{static/exampleList.ml}}
        \only<2>{\inputminted[firstline=2, lastline=2, linenos=false, frame=none, xleftmargin=0pt,]{ocaml}{static/exampleList.ml}}
        \only<3>{\inputminted[firstline=6, lastline=6, linenos=false, frame=none, xleftmargin=0pt,]{ocaml}{static/exampleList.ml}}
        \only<4>{\inputminted[firstline=7, lastline=7, linenos=false, frame=none, xleftmargin=0pt,]{ocaml}{static/exampleList.ml}}
      \end{minipage}
    };
  \end{tikzpicture}
\end{frame}



