\subsection{Analyse Lexicale}

\begin{frame}
    
    but: transformer le programme en entrée en une liste de lexèmes
    def lexème, bien faire la différence entre le lexème et le type reconnu, ie "Hello World" est un lexème de type string 

\end{frame}

%------------étude d'un exemple-----------
\begin{frame}
  \frametitle{Les automates: étude d'un exemple}
  \begin{tikzpicture}
    \begin{scope}
      [every node/.style={scale=0.6,font=\Large}, scale=0.4, initial text=]
      \tikzstyle{style0}=[draw=black]
      \tikzstyle{style1}=[draw=black]
      \tikzstyle{style2}=[draw=black]
      \tikzstyle{style3}=[draw=black]
      \tikzstyle{style4}=[draw=black]
      \tikzstyle{style5}=[draw=black]
      \tikzstyle{style6}=[draw=black]
      \tikzstyle{style7}=[draw=black]
      \tikzstyle{style8}=[draw=black]
      \tikzstyle{style9}=[draw=black]

      \only<1> {\tikzstyle{style0}=[draw=blue!50,very thick,fill=blue!20]}
      \only<2> {\tikzstyle{style1}=[draw=blue!50,very thick,fill=blue!20]}
      \only<3> {\tikzstyle{style2}=[draw=blue!50,very thick,fill=blue!20]}
      \only<4> {\tikzstyle{style3}=[draw=blue!50,very thick,fill=blue!20]}
      \only<5> {\tikzstyle{style4}=[draw=blue!50,very thick,fill=blue!20]}
      \only<6> {\tikzstyle{style5}=[draw=blue!50,very thick,fill=blue!20]}
      \only<7> {\tikzstyle{style0}=[draw=blue!50,very thick,fill=blue!20]}
      \only<8> {\tikzstyle{style8}=[draw=blue!50,very thick,fill=blue!20]}
      \only<9> {\tikzstyle{style0}=[draw=blue!50,very thick,fill=blue!20]}
      \only<10>{\tikzstyle{style9}=[draw=blue!50,very thick,fill=blue!20]}
      \only<11> {\tikzstyle{style0}=[draw=blue!50,very thick,fill=blue!20]}
      \only<12-23>{\tikzstyle{style6}=[draw=blue!50,very thick,fill=blue!20]}
      \only<24>{\tikzstyle{style7}=[draw=blue!50,very thick,fill=blue!20]}

      \node[state, style0, initial        ]  at (0,0) (0)          {$I$};

      \node[state, style1] (1) at (2,1) {1};
      \node[state, style2] (2) [right=0.4cm of 1] {2};
      \node[state, style3] (3) [right=0.4cm of 2] {3};
      \node[state, style4] (4) [right=0.4cm of 3] {4};
      \node[state, accepting right, style5] (5) [right=0.4cm of 4] {$F_1$};
      
      \node[state, style6,                ] (6) at (2,3) {5};
      \node[state, style7, accepting right] (7) [right=0.4cm of 6] {$F_2$};
      
      \node[state, style8, accepting right] (8) at (2,-1)      {$F_3$};
      \node[state, style9, accepting right] (9) at (2,-3)      {$F_4$};

      \path[->]
      (0) edge                node[above] {$*$} (8)
      (0) edge                node[above] {\LARGE$,$} (9)

      (0) edge                node[above] {$p$} (1)
      (1) edge                node[above] {$r$} (2)
      (2) edge                node[above] {$i$} (3)
      (3) edge                node[above] {$n$} (4)
      (4) edge                node[above] {$t$} (5)
      
      
      (0) edge                node[above] {$"$} (6)
      (6) edge [loop above]   node[above] {\large$[a-zA-Z0-9\_]$} (6)
      (6) edge                node[above] {$"$} (7)
      ;
      
    \end{scope}
    \node at (8, 0) (text) {\ttfamily print*,"Hello World"};
    \foreach \step in {1,2,...,6}{
      \node<\step>[right=\fpeval{\step * 5.75 - 3}pt of text.west, draw=blue, fill=blue!20, inner sep=0pt, minimum width=2pt, minimum height=10pt]{};
    }
    \foreach \step in {7,8}{
      \node<\step>[right=\fpeval{(\step - 1) * 5.75 - 3}pt of text.west, draw=blue, fill=blue!20, inner sep=0pt, minimum width=2pt, minimum height=10pt]{};
    }
    \foreach \step in {9,10}{
      \node<\step>[right=\fpeval{(\step - 2) * 5.75 - 3}pt of text.west, draw=blue, fill=blue!20, inner sep=0pt, minimum width=2pt, minimum height=10pt]{};
    }
    \foreach \step in {11,12,...,24}{
      \node<\step>[right=\fpeval{(\step - 3) * 5.75 - 3}pt of text.west, draw=blue, fill=blue!20, inner sep=0pt, minimum width=2pt, minimum height=10pt]{};
    }
    \node at (1, -2.3) (lb) {\texttt{res = [}};
    \only<7-> {\node [right] at (lb.east) (lb) {\texttt{'print'}};}
    \only<9-> {\node [right] at (lb.east) (lb) {\texttt{, '*'}};} 
    \only<11-> {\node [right] at (lb.east) (lb) {\texttt{, ','}};}
    \only<25> {\node [right] at (lb.east) (lb) {\texttt{, '"Hello World"'}};}
    \node [right] at (lb.east) {\texttt{]}};
  \end{tikzpicture}
\end{frame}

\begin{frame}
    exemple de la liste de lexèmes qui est donnée en Ocaml pour notre programme

\end{frame}


