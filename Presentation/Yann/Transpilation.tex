\section{La transpilation}

\begin{frame}
<<<<<<< Updated upstream
<<<<<<< Updated upstream
    \frametitle{Définition}
    \begin{tikzpicture}
    
    \tikzstyle{lien} = [->, >=latex]
    \tikzstyle{basic_text}=[text width=2cm, text badly centered]
    \tikzstyle{basic_node}=[draw = black,rounded corners=4pt, basic_text]
    \tikzstyle{wrapper}=[basic_node, inner sep=3pt]
    \tikzstyle{hidden}=[draw=black!0,color=black!0]
    \tikzstyle{faded}=[draw=black!20, color=black!20]
    

    \node [basic_node, text width=5cm, align=left] (Fortran) {
        \scalebox{0.8}{
            \parbox{\textwidth}{
                \textbf{Programe Fortran}
                \inputminted[firstline=1, lastline=3,firstnumber=1]{fortran}{static/HelloWorld.f90}
            }
        }
    };

    \node [basic_node, text width=5cm, align=left] (C) [right=1cm of Fortran] {
            \scalebox{0.8}{
            \parbox{\textwidth}{
                \textbf{Programe C}

        \inputminted[firstline=1, lastline=4,firstnumber=1]{c}{static/HelloWorld.c}
            } } 
    };
 
    \draw [lien] (Fortran.north) to[out=45, in=135]  node[midway, above] {transpilation}   (C.north west) ; 

\end{tikzpicture}
=======
=======
>>>>>>> Stashed changes
    \frametitle{Définition\esp}
    schéma avec un fichier Fortran, une machine et un fichier fortran
    et 
    programme fortran en entrée hello world et le programme obtenu en sortie 
>>>>>>> Stashed changes
\end{frame}


\begin{frame}
<<<<<<< Updated upstream
<<<<<<< Updated upstream

    \frametitle{Les étapes de la transpilation}

   \begin{center}

  \begin{tikzpicture}
    \tikzstyle{lien} = [->, >=latex]
    \tikzstyle{basic_text}=[text width=2cm, text badly centered, , minimum height=1cm]
    \tikzstyle{basic_node}=[draw = black,rounded corners=4pt, basic_text ]
    \tikzstyle{wrapper}=[basic_node, inner sep=3pt]

    
    \draw[white](-1.5,-4)rectangle(10,3);
      



    \node[basic_node] (A) at (0, 0) {analyse lexicale};
    \node[basic_node] (B) [right=0.5cm of A] {analyse syntaxique};
    \node[basic_node] (C) [right=0.5cm of B] {abstraction};

    \node[basic_node] (D) [right = 0.75cm of C] {traduction vers C};
    
    \draw [lien] (A) -- (B);
    \draw [lien] (B) -- (C);
    \draw [lien] (C) -- (D);
    
    
    \only{
      \node[basic_node] (E) [above = 0.75cm of D] {traduction vers Fortran};
      \node[basic_node] (F) [below = 0.75cm of D] {traduction vers Python};
      
      \node [wrapper, fit=(E), draw=red] (red_fit_2) {};
      \node [wrapper, fit=(F), draw=red] (red_fit_3) {};
      \draw [lien] (C) to [bend left] (E.west); 
      \draw [lien] (C) to [bend right] (F.west); 
    }<4->


    % \node[fit=(red_fit_1)(red_fit_2)(red_fit_3)(blue_fit)] (global_fit) {};
    
    \node<2-> [wrapper, fit=(A)(B)(C), draw=vertFonce] (blue_fit) {};
    \node<2-> [below=2cm of A.west, anchor=west, color=vertFonce] (label1) {module du langage d'entrée};
    
    \node<3-> [wrapper, fit=(D), draw=red] (red_fit_1) {};
    \node<3-> [below=0.5cm of label1.west, anchor=west, color=red] {modules des langages de sortie};

    
  \end{tikzpicture}
\end{center}
=======
=======
>>>>>>> Stashed changes
    \frametitle{\esp}
    parties d'un transpileur
    schéma tikz explication des différents paramètres (langage d'entrée et celui de sortie et le fait que ces langages sont constants pour le transpileur ) 
>>>>>>> Stashed changes
\end{frame}

