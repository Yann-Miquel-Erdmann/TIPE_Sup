\section{La transpilation}

\begin{frame}
<<<<<<< Updated upstream
<<<<<<< Updated upstream
    \frametitle{Définition}
    \input{tikz/FortranVersC.tex}
=======
=======
>>>>>>> Stashed changes
    \frametitle{Définition\esp}
    schéma avec un fichier Fortran, une machine et un fichier fortran
    et 
    programme fortran en entrée hello world et le programme obtenu en sortie 
>>>>>>> Stashed changes
\end{frame}


\begin{frame}
<<<<<<< Updated upstream
<<<<<<< Updated upstream

    \frametitle{Les étapes de la transpilation}

   \begin{tikzpicture}
  %placer les éléments
  \node[draw, rounded corners] (1) at (0,0) {Analyse lexicale Fortran};
  \node[draw, rounded corners] (2) [right=1cm of 1] {Analyse syntaxique Fortran};
  \node[fit=(1)(2), fill=briqueRouge, opacity=0.5, inner sep=0.1em, rounded corners] {};

  \node[draw,fill=vertdEau, rounded corners] (3) [right=1cm of 2] {Traduction Vers C};
  \node[draw,fill=vertdEau, rounded corners] (4) [above right =1cm of 2 ] {Traduction Vers Python};
  \node[draw,fill=vertdEau, rounded corners] (5) [below right =1cm of 2 ] {Traduction Vers Fortran};


  %les relier
  \draw[black, line width = 1pt, ->, >=latex] (1) -- (2);
  \draw[black, line width = 1pt, ->, >=latex] (2) -- (3);
  \draw[black, line width = 1pt, ->, >=latex] (2) -- (4);
  \draw[black, line width = 1pt, ->, >=latex] (2) -- (5);
  
\end{tikzpicture}

=======
=======
>>>>>>> Stashed changes
    \frametitle{\esp}
    parties d'un transpileur
    schéma tikz explication des différents paramètres (langage d'entrée et celui de sortie et le fait que ces langages sont constants pour le transpileur ) 
>>>>>>> Stashed changes
\end{frame}

