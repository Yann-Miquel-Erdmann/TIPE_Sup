\subsection{Analyse Syntaxique}

\begin{frame}
    but de l'analyse syntaxique, transformer la liste de lexèmes en un arbre de syntaxe pui un arbre de syntaxe abstraite
\end{frame}

\begin{frame}
    algo LL1
\end{frame}

%%%%%%%%%%%%%%%% pas pour Erwan
\begin{frame}
    tableau des first 
\end{frame}

\begin{frame}
    tableau des follow 
\end{frame}

\begin{frame}
    partie de pseudo code de LL1 ou sinon expliquer l'idée
\end{frame}

\begin{frame}
    on exécute l'algo sur hello world avec des \lin{\ pause}
    pour construire l'arbre de syntaxe
\end{frame}


\begin{frame}
    on montre le vrai arbre de syntaxe pour hello world
\end{frame}

\begin{frame}
    c'est pourquoi on transforme l'arbre de syntaxe en un arbre de syntaxe abstraite qui ne dépend pas du langage d'entrée  
    je dis que c'est Erwan qui l'a fait
    l'idée est de reconnaitre des motifs tels que l'assignation d'une variable ou bien l'affichage en console mais en oubliant la syntaxe utilisée  
    exemple de l'arbre de syntaxe en entrée et de l'ast en sortie  
\end{frame}
