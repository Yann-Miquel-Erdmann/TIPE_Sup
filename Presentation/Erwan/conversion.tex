\section{Traduction vers le langage de sortie}
\begin{frame}
  \frametitle{Traduction\esp}

  \begin{tikzpicture}
    % frame
    \node[frame] (frame) {};
    \tikzstyle{faded}=[draw=black!20,color=black!20]
    
    \node[below=4cm of frame.west, anchor=west, box, faded] (state_1) {\scriptsize Analyse Lexicale};
    \node[right=0.5cm of state_1.east, anchor=west, box, faded] (state_2) {\scriptsize Analyse Syntaxique};
    \node[right=0.5cm of state_2.east, anchor=west, box, faded] (state_3) {\scriptsize Syntaxe Abstraite};
    \node[right=0.5cm of state_3.east, anchor=west, box] (state_4) {\scriptsize Conversion};
    \draw[->, >=latex, faded] (state_1.east) -- (state_2.west);
    \draw[->, >=latex, faded] (state_2.east) -- (state_3.west);
    \draw[->, >=latex, faded] (state_3.east) -- (state_4.west);
    \draw[dashed] (frame.south west) -- (state_4.north west);
    \draw[dashed] (frame.south east) -- (state_4.north east);

    \node(def){parcours en profondeur};
    \node[below] at (def.south) {conversion en chaîne};

  \end{tikzpicture}

\end{frame}