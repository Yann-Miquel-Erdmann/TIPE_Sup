\section{Traduction vers le langage de sortie}

% TODO ajouter une diapo pour expliquer le but général (transformer l'arbre de syntaxe abstraite vers le langage de sortie)

\begin{frame}
  \frametitle{Principe général de la traduction}
  \begin{tikzpicture}
    \node (0, 0) {}; % to align the scope
    \begin{scope} [shift={(5.5, 0)}]
      \node [basic_node, text width=5.5cm, align=left] (Fortran) {
        \scalebox{0.8}{\textbf{Arbre de syntaxe abstraite}}
        \scalebox{0.9}{\begin{tikzpicture}
    \node[font=\scriptsize] (0) at (0, 0) {Programme};
    \node[font=\scriptsize] (1) at (-2, -1) {NouvelleLigne};
    \node[font=\scriptsize] (2) at (0, -1) {Print};
    \node[font=\scriptsize] (3) at (0, -2) {'"Hello World"'};
    \node[font=\scriptsize] (4) at (2, -1) {NouvelleLigne};
    
    \draw (0) -- (1);
    \draw (0) -- (2);
    \draw (2) -- (3);
    \draw (0) -- (4);
\end{tikzpicture}}
      };

      \node [basic_node, text width=5.5cm, align=left, below=1cm of Fortran] (C) {
        \scalebox{0.8}{
          \parbox{\textwidth}{
            \textbf{Programme C}
            \inputminted{c}{static/HelloWorld.c}
          }
        } 
      };

      \draw [lien, thick] (Fortran.south) to (C.north);
    \end{scope}
  \end{tikzpicture}
\end{frame}

\begin{frame}
  \frametitle{Traduction\esp}

  \begin{tikzpicture}
    % frame
    \node[frame] (frame) {};
    \tikzstyle{faded}=[draw=black!20,color=black!20]
    
    \node[below=4cm of frame.west, anchor=west, box, faded] (state_1) {\scriptsize Analyse lexicale};
    \node[right=0.5cm of state_1.east, anchor=west, box, faded] (state_2) {\scriptsize Analyse syntaxique};
    \node[right=0.5cm of state_2.east, anchor=west, box, faded] (state_3) {\scriptsize Abstraction};
    \node[right=0.5cm of state_3.east, anchor=west, box] (state_4) {\scriptsize Conversion};
    \draw[->, >=latex, faded] (state_1.east) -- (state_2.west);
    \draw[->, >=latex, faded] (state_2.east) -- (state_3.west);
    \draw[->, >=latex, faded] (state_3.east) -- (state_4.west);
    \draw[dashed] (frame.south west) -- (state_4.north west);
    \draw[dashed] (frame.south east) -- (state_4.north east);

    \node(def) at (0, 0.5) {parcours en profondeur de l'AST};
    \node[below] at (def.south) {conversion en chaîne};

  \end{tikzpicture}

\end{frame}