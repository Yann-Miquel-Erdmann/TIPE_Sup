\section{Analyse syntaxique}
\begin{frame}
  \frametitle{Principe général de l'analyse syntaxique}
  \begin{tikzpicture}
    \node (0, 0) {}; % to align the scope
    \begin{scope} [shift={(5.5, 0)}]
      \node [basic_node, text width=10.5cm, align=left] (Fortran) {
        \scalebox{0.8}{
          \parbox{\textwidth}{
            \textbf{Liste de lexèmes}
            \footnotesize \inputminted[frame=none, xleftmargin=0pt, linenos=false, breaklines=false]{ocaml}{static/listeLexemes.ml}
          }
        } 
      };

      \node [basic_node, text width=10cm, below=1cm of Fortran, align=left] (C) {
        \scalebox{0.8}{\textbf{Arbre de syntaxe}}
        \scalebox{0.85}{    
\tikzstyle{feuille}=[ draw, rectangle, inner sep = 0.12cm]
\tikzstyle{noeud}=[ draw, rectangle,rounded corners, minimum width= 0.64cm, line width = 1pt]
\begin{tikzpicture}[
        baseline=(base), 
        level/.style={sibling distance = 1.6cm/#1, level distance = 1cm},
        every node/.style={scale=0.6, font=\footnotesize}
    ]
    
    \node[ noeud] {Programme}
    child { 
        node[noeud] (base) {ProgramMC}
    }
    child {
        node[noeud] {NomProg}
        child {node[feuille] {"hello"}}
    }
    child [ level distance=2cm]{ 
        node [ noeud] {Print}
        child [sibling distance= 1cm] {
            node [noeud] {PrintMC}
        }
        child [sibling distance= 1cm]{
            node [noeud] {Asterisque}
        }
        child [sibling distance= 1cm]{
            node [noeud] {Virgule}
        }
        child [sibling distance= 1cm]{
            node [noeud] {Chaine}
            child { 
                node [feuille] {"Hello World"}
            }
        }
        child [sibling distance= 1cm] {
            node [noeud] {Paramliste}
            child { 
                node [feuille] {$\varepsilon$}
            }
        }
    }
    child { 
        node[ noeud] {EndProgramMC}
    }
    child {
        node[noeud] {NomProg}
        child {node[feuille] {"hello"}}
    };

\end{tikzpicture}}
      };

      \draw [lien, thick] (Fortran.south) to (C.north);
    \end{scope}
  \end{tikzpicture}
\end{frame}

\tikzstyle{box}=[draw, rounded corners=4pt]
\tikzstyle{frame}=[box, minimum width=11.3cm, minimum height=6cm]
\tikzstyle{faded}=[color=black!25]
\tikzstyle{hidden}=[draw=white!0, color=white!0]

%------------grammaire-----------
\subsection{La grammaire}
\begin{frame}
  \frametitle{Analyse syntaxique\esp}

  \begin{tikzpicture}
    % frame
    \node[frame] (frame) {};

    \node[below=0.5cm of frame.north west, anchor=north west] (1) {\point Grammaire};
    \node[faded, below=0.6cm of 1.west, anchor=west] (_2) {\point LL1};

    \tikzstyle{faded}=[draw=black!20,color=black!20]
    \node[below=4cm of frame.west, anchor=west, box, faded] (state_1) {\scriptsize Analyse lexicale};
    \node[right=0.5cm of state_1.east, anchor=west, box] (state_2) {\scriptsize Analyse syntaxique};
    \node[right=0.5cm of state_2.east, anchor=west, box, faded] (state_3) {\scriptsize Abstraction};
    \node[right=0.5cm of state_3.east, anchor=west, box, faded] (state_4) {\scriptsize Conversion};
    \draw[->, >=latex, faded] (state_1.east) -- (state_2.west);
    \draw[->, >=latex, faded] (state_2.east) -- (state_3.west);
    \draw[->, >=latex, faded] (state_3.east) -- (state_4.west);
    \draw[dashed] (frame.south west) -- (state_2.north west);
    \draw[dashed] (frame.south east) -- (state_2.north east);

    % définition 
    \only<1->{
      \node[right=4.2cm of 1.east, anchor=west] (def) {$\mathcal{G}=(\Sigma, V, P, \mathcal{S})$};
      \node[below=0.25cm, font=\ttfamily] (def1) at (def) {$R\in V\rightarrow W_1...W_n \in (V\times\Sigma)^{n}$};
    }

    % exemple simple
    \only<2->{
      \node[below=0.2 cm of def1] {Exemple:};
      \node[below=1.4cm of def.200, anchor = north west] {
        \begin{minipage}{0.5\textwidth}
          \inputminted[linenos=true]{text}{static/Grammaire_example.txt}
        \end{minipage}
      };
      \node[below=4.4cm of def.200, font=\scriptsize, anchor=west] {Grammaire de $a^* \cdot b^*$};
    }

    % application à une chaine simple
    \only<3->{
      \begin{scope}[shift={(-2.5, 1.8)}, font=\footnotesize]
        \tikzstyle{style1}=[hidden]
        \tikzstyle{style2}=[hidden]
        \tikzstyle{style3}=[hidden]
        \tikzstyle{style4}=[hidden]
        \tikzstyle{style5}=[hidden]

        \only<4->{\tikzstyle{style1}=[font=\ttfamily]}
        \only<5->{\tikzstyle{style2}=[font=\ttfamily]}
        \only<6->{\tikzstyle{style3}=[font=\ttfamily]}
        \only<7->{\tikzstyle{style4}=[font=\ttfamily]}
        \only<8->{\tikzstyle{style5}=[font=\ttfamily]}

        \node [font=\ttfamily] (0) at (0, 0) {S};
        \node [below=0.3cm of 0, style1] (1) {A};
        \node [below=0.3cm of 1.south west, style2] (2) {$a$};
        \node [below=0.3cm of 1.south east, style2] (3) {A};
        \node [below=0.3cm of 3, style3] (4) {B};
        \node [below=0.3cm of 4.south west, style4] (5) {$b$};
        \node [below=0.3cm of 4.south east, style4] (6) {B};
        \node [below=0.3cm of 6, style5] (7) {$\varepsilon$};

        \draw [style1] (0) -- (1);
        \draw [style2] (1) -- (2);
        \draw [style2] (1) -- (3);
        \draw [style3] (3) -- (4);
        \draw [style4] (4) -- (5);
        \draw [style4] (4) -- (6);
        \draw [style5] (6) -- (7);
      \end{scope}
    }

  \end{tikzpicture}

\end{frame}

%-------------LL1--------------
\subsection{L'algorithme LL1}
\begin{frame}
  \begin{tikzpicture}
    % frame
    \node[frame] (frame) {};

    \node[below=0.5cm of frame.north west, anchor=north west, faded] (1) {\point Grammaire};
    \node[below=0.6cm of 1.west, anchor=west] (_2) {\point LL1};

    \tikzstyle{faded}=[draw=black!20,color=black!20]
    \node[below=4cm of frame.west, anchor=west, box, faded] (state_1) {\scriptsize Analyse lexicale};
    \node[right=0.5cm of state_1.east, anchor=west, box] (state_2) {\scriptsize Analyse syntaxique};
    \node[right=0.5cm of state_2.east, anchor=west, box, faded] (state_3) {\scriptsize Abstraction};
    \node[right=0.5cm of state_3.east, anchor=west, box, faded] (state_4) {\scriptsize Conversion};
    \draw[->, >=latex, faded] (state_1.east) -- (state_2.west);
    \draw[->, >=latex, faded] (state_2.east) -- (state_3.west);
    \draw[->, >=latex, faded] (state_3.east) -- (state_4.west);
    \draw[dashed] (frame.south west) -- (state_2.north west);
    \draw[dashed] (frame.south east) -- (state_2.north east);

    % petite description de LL1
    \node[font=\ttfamily] (list) at (-4, 0){[a, b]};
    \node[right] (+) at (list.east) {et};
    \node[right] at (+.east) {
        \begin{minipage}{0.5\textwidth}
          \inputminted{text}{static/Grammaire_example.txt}
        \end{minipage}
    };
    \only<2>{
      \node[font=\ttfamily] at (0.5, 0) {$\Rightarrow$};
      \begin{scope}[shift={(2.5, 1.8)}, font=\footnotesize]
        
        \node [font=\ttfamily] (0) at (0, 0) {S};
        \node [below=0.3cm of 0, font=\ttfamily] (1) {A};
        \node [below=0.3cm of 1.south west, font=\ttfamily] (2) {$a$};
        \node [below=0.3cm of 1.south east, font=\ttfamily] (3) {A};
        \node [below=0.3cm of 3, font=\ttfamily] (4) {B};
        \node [below=0.3cm of 4.south west, font=\ttfamily] (5) {$b$};
        \node [below=0.3cm of 4.south east, font=\ttfamily] (6) {B};
        \node [below=0.3cm of 6] (7) {$\varepsilon$};

        \draw (0) -- (1);
        \draw (1) -- (2);
        \draw (1) -- (3);
        \draw (3) -- (4);
        \draw (4) -- (5);
        \draw (4) -- (6);
        \draw (6) -- (7);
      \end{scope}
    }
    
  \end{tikzpicture}

\end{frame}