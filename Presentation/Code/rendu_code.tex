\documentclass[12pt,a4paper,fleqn]{article}

%%%%%%%%%%%%%%%%%%%%%% LES PACKAGES
\input{packages_MPI.tex}

%%%%%%%%%%%%%%%%%%%%%% MISE EN FORME
\colorlet{expli}{gray}
\input{mise_en_forme_MPI_Yann.tex}

%%%%%%%%%%%%%%%%%%%%%%%%%%%%%%%%%%%%%%%
\begin{document}
\pagestyle{fancy} %active les pieds de pages


%%%%%%%%%%%%%%%%%%%%%%%%%%%%%%%%%%%%%%%
\titreRenduCode{La transpilation de Fortran vers C}{Juillet 2025}

\tableofcontents
\newpage

%%%%%%%%%%%%%%%%%%%%%%%%%%%%%%%%%%%%%%%
\section{Grammaire et exemples}
\subsection{Grammaire du Fortran}
\inputminted{ocaml}{../../grammar/Our_Grammar.txt}
\newpage
\tiny
\textcolor{white}{test}
\normalsize

\subsection{Exemple Fibonacci}
\inputminted{ocaml}{../../tests/Fortran/fibonacci.f90}
\inputminted{c}{../../tests/C/fibonacci.c}
\newpage

\subsection{Exemple conversion binaire}
\inputminted{ocaml}{../../tests/Fortran/fibonacci.f90}
\inputminted{c}{../../tests/C/fibonacci.c}
\newpage


\section{Code du pré-calcul}
%======================================


\foreach \name in{src/preprocessing/generateMlFiles.ml, src/prebuild/buildAutomaton.ml}{
    \subsection{\name}
    \inputminted{ocaml}{../../\name}
    \newpage
    \tiny
    \textcolor{white}{test}
    \normalsize
}


\section{Code du transpileur}
\foreach \name in{src/abstractTokens.ml, src/bibliotheques.ml, src/environnement.ml, src/symbols.ml, src/traduction.ml, src/regex.ml, src/grammarFunctions.ml, src/grammar.ml, src/automates.ml, src/LL1.ml, src/convertToAbstract.ml,  src/traductionC.ml, src/traductionFortran.ml, src/transpileurs.ml, src/main.ml}{
    \subsection{\name}
    \inputminted{ocaml}{../../\name}
    \newpage
    \tiny
    \textcolor{white}{test}
    \normalsize
}


\end{document}
